\documentclass{article}
\usepackage{graphicx} % Required for inserting images
\usepackage{amsthm, amssymb,amsmath}

\title{Homework 4}
\author{Bryan Lee}
\date{October 10, 2024}

\begin{document}

\maketitle

\noindent {\bf 1.} Recall that an $n \times n$ matrix $Q$ is orthogonal if the
columns of $Q$ form an orthonormal basis of $\mathbb{R}^n$. This is equivalent to
$Q^TQ = QQ^T = I_n$. \\
{\bf a)} Show that the product of two $n \times n$ orthogonal matrices is an
orthogonal matrix. \\

\noindent Let $Q = Q_1Q_2$, then we have
\[
Q^T = (Q_1Q_2)^T = Q_2^TQ_1^T
\]
Now, to verify orthogonality (multiplying both sides with Q):\\
\[
Q^TQ = (Q_2^TQ_1^T)(Q_1Q_2)
\]
\[
Q^TQ = Q_2^T(Q_1^TQ_1)Q_2
\]
\[
Q^TQ = Q_2^TI_nQ_2
\]
\[
Q^TQ = I_n
\]
Thus, the product of the two $nxn$ orthogonal matrices is an orthogonal matrix because $Q^TQ = I_n$ where Q is the product of the two $nxn$ orthogonal matrices.\\\\

\noindent {\bf b)} Prove that if $Q$ is an orthogonal matrix, so is $Q^T$. Deduce that the
rows of an orthogonal matrix also form an orthonormal basis. \\

\noindent Suppose that $Q^T$ is an orthogonal matrix, then:
\[
(Q^T)^TQ^T = QQ^T = I_n \Leftrightarrow Q^T(Q^T)^T = Q^TQ = I_n
\]
This shows that if Q is an orthogonal matrix, so is $Q^T$. We know that if Q is orthogonal, then its columns form an orthonormal basis for $\mathbb{R}^n$. Now, since we have just proven that $Q^T$ is also orthogonal, we can consider the columns of $Q^T$. The columns of $Q^T$ are exactly the rows of Q, due to transpose swapping the rows and columns. Thus, the rows of Q also form an orthonormal basis of $\mathbb{R}^n$.\\\\

\noindent {\bf c)} Show that $\begin{pmatrix} \cos \theta & - \sin \theta \\ \sin \theta & \
cos \theta \end{pmatrix}$ is an orthogonal matrix for any $0 \leq \theta < 2\pi$.
What does this matrix do? \\

\[
Let\hspace{0.1cm}Q =
\begin{pmatrix}
cos\theta & -sin\theta\\
sin\theta & cos\theta\\
\end{pmatrix}
\]
Now, to show that it is orthogonal:
\[
i)\hspace{0.1cm}Q^T =
\begin{pmatrix}
cos\theta & sin\theta\\
-sin\theta & cos\theta\\
\end{pmatrix}
\]
\[
ii)\hspace{0.1cm}Q^TQ =
\begin{pmatrix}
cos\theta & sin\theta\\
-sin\theta & cos\theta\\
\end{pmatrix}
\begin{pmatrix}
cos\theta & -sin\theta\\
sin\theta & cos\theta\\
\end{pmatrix} =
\]
\[
\begin{pmatrix}
cos\theta cos\theta + sin\theta sin\theta & cos\theta (-sin\theta) + sin\theta cos\theta \\
(-sin\theta) cos\theta + cos\theta sin\theta & (-sin\theta) (-sin\theta) + cos\theta cos\theta
\end{pmatrix} = 
\]
\[
\begin{pmatrix}
cos^2\theta + sin^2\theta & 0\\
0 & sin^2\theta + cos^2\theta
\end{pmatrix}
=
\begin{pmatrix}
1 & 0 \\
0 & 1 \\
\end{pmatrix}
=
I_n
\]
Thus, $Q = \begin{pmatrix} \cos \theta & - \sin \theta \\ \sin \theta & \
cos \theta \end{pmatrix}$ is an orthogonal matrix for any $0 \leq \theta < 2\pi$ as $Q^TQ = I_n$. The matrix represents a rotational matrix. If we apply Q to a vector v, it will rotate it  counterclockwise by $\theta$.\\
\[
Qv =
\begin{pmatrix}
cos\theta & -sin\theta\\
sin\theta & cos\theta\\
\end{pmatrix}
\begin{pmatrix}
x \\
y \\
\end{pmatrix}
=
\begin{pmatrix}
xcos\theta - ysin\theta\\
xsin\theta + ycos\theta\\
\end{pmatrix}
\]\\
For example, taking vector
\[
v = 
\begin{pmatrix}
1\\
2\\
\end{pmatrix}
\]
and flipping it across the y-axis by rotating v by $\theta = \pi$\\
\[
\begin{pmatrix}
cos\pi & -sin\pi\\
sin\pi & cos\pi\\
\end{pmatrix}
\begin{pmatrix}
1 \\
2 \\
\end{pmatrix}
=
\begin{pmatrix}
-1\\
-2\\
\end{pmatrix}
\]\\\\

\noindent {\bf d)} Prove that if $Q$ is an $n \times n$ orthogonal matrix then $||Q {\bf x}
|| = ||{\bf x}||$ for any ${\bf x} \in \mathbb{R}^n$. \\

\noindent The norm of a vector $x \in \mathbb{R}^n$ is defined as:
\[
||x|| = \sqrt{x^Tx}
\]
\[
||Qx|| = \sqrt{(Qx)^T(Qx)}
\]
Note: $(Qx)^T \Leftrightarrow x^TQ^T$
\[
||Qx|| = \sqrt{x^TQ^TQx}
\]
\[
||Qx|| = \sqrt{x^Tx} = ||x||
\]
Thus, if Q is an $nxn$ orthogonal matrix then $||Qx|| = ||x||$ for any $x \in \mathbb{R}^n$.\\\\

\noindent {\bf 2c)} Just to be implicitly clear: The answer to part 2c is above the myHouseholder function as a print statement in the MATLAB code.

\end{document}
