\documentclass[12pt]{article}
\usepackage{amsmath, amssymb, amsthm, graphicx, geometry, listings, xcolor}
\geometry{margin=1in}
\definecolor{darkgreen}{rgb}{0,0.5,0}

\lstset{
    language=Matlab,                 % Language of the code
    basicstyle=\ttfamily,            % Code font style
    keywordstyle=\color{blue},       % Keywords color
    commentstyle=\color{darkgreen},  % Comments color
    stringstyle=\color{red},         % Strings color
    breaklines=true,                 % Automatic line breaking
    numbers=left,                    % Line numbers on the left
    frame=single                     % Adds a border around the code
}

\title{Computation of the Camera Matrix}
\author{Bryan Lee}
\date{December 13, 2024}

\begin{document}

\maketitle

\section{Introduction}
In computer vision, the camera matrix P is a 3x4 matrix that defines the mathematical transformation mapping of 3-D points in world space into their corresponding 2-D coordinates on the camera's image plane. Essentially, it models how a camera perceives a scene by incorporating both the intrinsic camera parameters (such as focal length, principle point, skew, and distortion scale) and extrinsic parameters (position and orientation) required for this transformation.

\section{Preliminaries}
A standard camera matrix P takes the form:
\[
	\begin{array}{rrcl}
        x = PX & \Leftrightarrow & x = KR[I_3 | -X_O]X
	\end{array}
\]
Where \( x \) are the observed image points (2-D), X is the given control coordinates in the world space (3-D), K is the intrinsic (calibration) 3x3 matrix, R is the 3x3 rotation matrix, \( [I_3 | -X_O] \) represents the 3x4 extrinsic transformation (combining rotation and translation), and finally, \( X_O \) is the 3x1 translation matrix that represents the camera's position relative to the object.\\

\begin{figure}[ht]
    \centering
    \includegraphics[width=0.5\linewidth]{PrelimCameraMatrixDiagram.png}
    \caption{A simple diagram of the camera matrix: x = PX.}
\end{figure}

\noindent To compute the said camera matrix P, we will compute a linear solution:
\begin{enumerate}
    \item Compute an initial estimate of \( P \) by a linear approach: \begin{enumerate}
        \item Use a similarity transformation matrix \( T \) and \( U \) to normalize both image points and world space.
        \item Use direct linear transformation (DLT) to form a \( 2n \times 12 \) matrix \( A \) by stacking each corresponding normalized image and world points. The vector \( p \) will then contain the entries of the matrix \( \tilde{P} \). A solution of \( Ap = 0 \), with \( ||p|| = 1 \), is obtained from the singular vector with the smallest singular value.
    \end{enumerate}

    \item Denormalize the normalized camera matrix \( \tilde{P} \).
\end{enumerate}

\noindent \textbf{Note:} Point correspondences mean \( \{ X_i \leftrightarrow x_i \} \).\\

One observation that we need to note is that the current DLT approach assumes an affine camera model. This means that the projection matrix only involves scaling, rotation, and translation without any nonlinear effects like perspective. DLT also solves for uncalibrated cameras, where the camera's intrinsic and extrinsic parameters are unknown, with the obvious consequence of losing accuracy.

In the case of an uncalibrated camera, the projection matrix \( P \) contains 11 unknown parameters: nine from the intrinsic camera matrix \( K \) and three from the extrinsic parameters (the rotation matrix \( R \) and the translation vector \( t \)). These unknowns need to be solved simultaneously using the point correspondence between 3-D world points and their 2-D image projections.

Using \( n \geq 6 \) corresponding world and image points, we can simultaneously solve the unknowns and mitigate geometric errors. Each point correspondence contributes two equations (for the x and y image coordinates), so with 6 points, we can acquire 12 equations, which is enough to solve for the 11 unknowns in \( P \).

\section{Normalization of the image and world points}
The first step to computing the transformation camera matrix \( P \) is to normalize the image points with their respective similarity transformation matrix \( T\). We aim to transform the points \( x_i \) into a new set of points \( \tilde{x}_i \) such that the centroid of \( \tilde{x}_i \) is shifted to the origin \( (0, 0)^\top \), and then scaled so that their average distance from the origin becomes \( \sqrt{2} \). Refer to Figure 2 below for more clarification. \\

\noindent Given a set of 2-D image points (x, y):
\[
    x = \{(x_1, y_1), ..., (x_n, y_n)\}
\]
The centroid of the points is calculated as:
\[
        (\bar{x}, \bar{y}) = ( \frac{1}{n}\sum_{i=1}^{n} x_i,  \frac{1}{n}\sum_{i=1}^{n} y_i)
        \text{, in other words: taking the mean of the points.}
\]
Next, we compute the new set of points \( \tilde{x_i} \) such that the centroid of the points \( \tilde{x_i} \) is the coordinate origin \( (0, 0)^T \):
\[
    \tilde{x}_i = x_i - \bar{x}, \quad \tilde{y}_i = y_i - \bar{y}
\]
The points are scaled to make the average distance from the origin equal to \( \sqrt{2} \). Let the distance of a point from the origin be:
\[
    d_i = \sqrt{\tilde{x}_i^2 + \tilde{y}_i^2}
\]
The scaling factor \( s \) is then computed as:
\[
    s = \frac{\sqrt{2}}{\frac{1}{n} \sum_{i=1}^{n} d_i}
\]
Next, we create the image point transformation matrix T:
\[
T = \begin{bmatrix}
\text{s} & 0 & -\text{s} \cdot \bar{x} \\
0 & \text{s} & -\text{s} \cdot \bar{y} \\
0 & 0 & 1
\end{bmatrix}
\]
Finally, we normalize the 2-D image points with the similarity transformation T matrix:
\[
x_{\text{norm}} = (T \cdot x^T)^T
\]

\noindent We will need to repeat the process for a set of 3-D world-space points (x, y, z) such that the average distance from the origin is now \( \sqrt{3} \) and similarly the U matrix and \( X_{norm} \) is:
\[
U = \begin{bmatrix}
\text{s} & 0 & 0 & -\text{s} \cdot \bar{x} \\
0 & \text{s} & 0 & -\text{s} \cdot \bar{y} \\
0 & 0 & \text{s} & -\text{s} \cdot \bar{z} \\
0 & 0 & 0 & 1
\end{bmatrix}
\]
\[
X_{\text{norm}} = (U \cdot X^T)^T
\]
\newpage

\begin{figure}[ht]
    \centering
    \includegraphics[width=0.75\linewidth]{NormalizationFigure.png}
    \caption{The blue dots are the given set of 2-D image points with their respective centroid \( (\bar{x}, \bar{y}) \). The red dots are the image points with their centroid shifted to the origin. The green dots are just additional translations that we will not need. }
\end{figure}

\section{Computing the direct linear transformation (DLT)}

\noindent The second step in computing the P camera matrix involves using the Direct Linear Transformation (DLT) algorithm to calculate the normalized camera projection matrix.
P.
\[ 
X_{i} = P_{3 \times 4} X_{i} = 
\begin{pmatrix}
P_{11} & P_{12} & P_{13} & P_{14} \\
P_{21} & P_{22} & P_{23} & P_{24} \\
P_{31} & P_{32} & P_{33} & P_{34}
\end{pmatrix} X_{i}
\]

\noindent Let:
\[ 
A = 
\begin{pmatrix}
P_{11} \\ P_{12} \\ P_{13} \\ P_{14}
\end{pmatrix}
B = 
\begin{pmatrix}
P_{21} \\ P_{22} \\ P_{23} \\ P_{24}
\end{pmatrix}
C = 
\begin{pmatrix}
P_{31} \\ P_{32} \\ P_{33} \\ P_{34}
\end{pmatrix}
\]

\noindent Rearrange:
\[
\begin{pmatrix}
U_{i} \\ V_{i} \\ W_{i}
\end{pmatrix} = x_{i}=PX_{i} = 
\begin{pmatrix}
A^{T} \\ B^{T} \\ C^{T}
\end{pmatrix} x_{i}
\]\\

\noindent 
P = ($A^{T} \times X_{i}, B^{T} \times x_{i}, C^{T} \times x_{i}$) \\

\noindent
$P_{c} = (u_{i}, v_{i})$, we can use a similar triangle:
$\frac{f}{C^{T} \times x_{i}} = \frac{u_{i}}{A^{T} \times X_{i}} = \frac{v_{i}}{B^{T} \times x_{i}}$ \\

\noindent giving us: \\
$x_{i} = \frac{u_{i}}{w_{i}} = \frac{A^{T} \times x_{i}}{C^{T} \times x_{i}}$,
$y_{i} = \frac{v_{i}}{w_{i}} = \frac{B^{T} \times x_{i}}{C^{T} \times x_{i}}$\\

\noindent for\\
\[x_{1} = 
\begin{pmatrix}
x_{1}\\
y_{i}\\
1
\end{pmatrix} = 
\begin{pmatrix}
u_{1}\\
v_{i}\\
w_{i}
\end{pmatrix} = 
\begin{pmatrix}
A^{T} \times X_{i}\\
B^{T} \times X_{i}\\
C^{T} \times X_{i}
\end{pmatrix}
\]

\noindent Thus, we obtain the following.\\
$x_{i}C^{T}x_{i}-A^{T}x_{i} = 0, \\
and \\
y_{i}C6{T}x_{i} - B^{T}X_{i} = 0$\\

\noindent To get:
\[
\left\{
    \begin{array}{l}
        -x_{i}^{T}A + x_{i}x_{i}^{T}C = 0 \\
        -x_{i}^{T}B + y_{i}x_{i}^{T}C = 0
    \end{array}
\right.
\]
\end{document}
