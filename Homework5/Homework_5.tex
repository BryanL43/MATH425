\documentclass{article}
\usepackage{amsthm, amssymb,amsmath}
\usepackage{graphicx} % Required for inserting images

\title{Homework 5}
\author{Bryan Lee}
\date{November 9, 2024}

\begin{document}

\maketitle

\noindent
{\bf 1.} Find the closest point from ${\bf b} = \begin{pmatrix} 1 \\ 1 \\ 2 \\-2 \end{pmatrix}$ to the subspace spanned by
$$\begin{pmatrix} 1 \\ 2 \\ -1 \\0 \end{pmatrix}, \,\, \begin{pmatrix} 0 \\ 1 \\ -
2 \\ -1 \end{pmatrix}, \,\, \begin{pmatrix} 1 \\ 0 \\ 3 \\ 2 \end{pmatrix}.$$
Use any {\tt MATLAB} command that you think is useful to do this computation. \\

\noindent {\tt Some further explanation on my computation on MATLAB:}\\
Since there is no solution to the original problem of $rref([A | b])$ where A is the matrix with the three augmented vectors. We need to compute the least square solution. To do that, I compute via row operations $A^T Ax = A^T b$ where $A^T A$ becomes a symmetric matrix and $A^T b$ will be a vector.

\[
\hat{A} = A'A = 
\begin{pmatrix}
6 & 4 & -2\\
4 & 6 & -8\\
-2 & -8 & 14
\end{pmatrix}
\]
\[
\hat{b} = A'b = 
\begin{pmatrix}
1\\
-1\\
3
\end{pmatrix}
\]
\[
rref([\hat{A} | \hat{b}]) = 
\left( \begin{array}{ccc|c}
6 & 4 & -2 & 1 \\
4 & 6 & -8 & -1 \\
-2 & -8 & 14 & 3 \\
\end{array} \right) = 
\left( \begin{array}{ccc|c}
1 & 0 & 1 & \frac{1}{2} \\
0 & 1 & -2 & -\frac{1}{2} \\
0 & 0 & 0 & 0 \\
\end{array} \right)
\]\\
The rref shows that there is a free variable for the third unknown variable. We can assign that variable, let's say, $z$, as an arbitrary value $t$. Thus, we get the three equations: $x = \frac{1}{2} - t$, $y = -\frac{1}{2} + 2t$, and $z = t$. Now, putting the equation together as a vector:\\
\[
v = 
\begin{pmatrix}
\frac{1}{2} - t\\
-\frac{1}{2} +2t\\
t
\end{pmatrix} = 
\begin{pmatrix}
\frac{1}{2}\\
-\frac{1}{2}\\
0
\end{pmatrix} + t
\begin{pmatrix}
-1\\
2\\
1
\end{pmatrix}
\]\\
Since t is an arbitrary value and z is a linear combination of the previous two vectors, we can take the basis vector and multiply it with A, giving us the closest points:
\[
A * v = 
\begin{pmatrix}
1 & 0 & 1\\
2 & 1 & 0\\
-1 & -2 & 3\\
0 & -1 & 2
\end{pmatrix}
\begin{pmatrix}
\frac{1}{2}\\
-\frac{1}{2}\\
0
\end{pmatrix} = 
\begin{pmatrix}
\frac{1}{2} \\
\frac{1}{2} \\
\frac{1}{2} \\
\frac{1}{2}
\end{pmatrix}
\]\\
Running MATLAB A\slash b yields the same solution.
\\\\

\noindent{\bf 4.} Let $f(x) = x^2$ on the interval $[0,2\pi]$. In this exercise, you will
compute the discrete Fourier coefficients $c_0, c_1, \ldots, c_7$ of $f$ from the
sample vector
$$ {\bf f} \, = \, \begin{pmatrix} f_0 \\ f_1 \\ \vdots \\ f_7 \end{pmatrix}$$
where $f_j = f(j2\pi/8)$ for $j=0,\ldots, 7$. \\\\

\noindent Part d) Explanation of computing $p(x)$ is explained as a comment in the MATLAB code.\\

\noindent{\bf e)} Find out how you can plot the graph of a function in {\tt MATLAB}. Then
plot the graphs of $f(x)$ and $p_1(x)$ on the interval $[0,2\pi]$. What do you see?\\

\noindent Graph is rendered on Matlab code.\\
Analysis of the $p_1(x)$ graph:\\
The graph has a sinusoidal wave that has a few points that agree with $f(x) = x^2$. However, it does have a lot of "noise" and roughly follows the original $f(x)$ line. One peculiar aspect of the $p_1(x)$ graph is the sinusoidal wave alternation from being above the original $x^2$ to following it and ending up below it as it approached $2\pi$. Another noteworthy aspect of the $p_1(x)$ graph is the sudden drop off at $7\pi/4$ from following the $x^2$ line to 0. Overall, the graph approximately follows the original signal line, but only by agreeing at specific points.\\\\

\noindent {\bf h)} Plot the graph of $f(x)$ and $q_1(x)$ on the interval $[0,2\pi]$. Now, what
do you observe?\\

\noindent Graph is rendered on Matlab code.\\
Analysis of the $q_1(x)$ graph:\\
The graph is still a tiny bit wavy, but it follows the original $x^2$ graph much more closely. Especially between the points $\pi/2$ and $3\pi/2$, the $q_1(x)$ graph hovers/on the $x^2$ line. The points between 0 and $\pi/2$ also seem to be a shifted sinusoidal wave before seemingly being forced to wrap on the $x^2$ graph afterward. There is still a sudden drop off at around $7\pi/4$ from following the $x^2$ line to 0. Overall, the $q_1(x)$ graph better represents the $x^2$ graph but is still only an approximation that is more accurate than $p_1(x)$.

\end{document}
